% \iffalse meta-comment
%
% sidenotes.dtx
%
% Copyright (C) 2011-2020 by Andy Thomas <andythomas(at)web.de>
%
% This work may be distributed and/or modified under the
% conditions of the LaTeX Project Public License, either version 1.3
% of this license or (at your option) any later version.
% The latest version of this license is in
% http://www.latex-project.org/lppl.txt
% and version 1.3 or later is part of all distributions of LaTeX
% version 2003/12/01 or later.
% 
% The author of this work is Andy Thomas
%
%<*driver>
\ProvidesFile{caesar.dtx}[2020/04/05 v0.20 rich text in the margin for LaTeX]%
%</driver>
%<package>\RequirePackage{l3keys2e}%
%<package>\ProvidesExplPackage{caesar}{2020/04/05}{0.20}{rich text in the margin for LaTeX}
%<package>\RequirePackage{caption} % handles the captions (in the margin)
%<package>\RequirePackage{xparse} % new LaTeX3 syntax to define macros and environments
%<package>\RequirePackage[strict]{changepage} % Changepage package for symmetric twoside handling
%<*driver>
\documentclass{ltxdoc}
\CodelineIndex
\RecordChanges
\begin{document}
  \DocInput{caesar.dtx}
\PrintIndex
  \PrintChanges
\end{document}
%</driver>
% \fi
%
%  \CharacterTable
%  {Upper-case    \A\B\C\D\E\F\G\H\I\J\K\L\M\N\O\P\Q\R\S\T\U\V\W\X\Y\Z
%   Lower-case    \a\b\c\d\e\f\g\h\i\j\k\l\m\n\o\p\q\r\s\t\u\v\w\x\y\z
%   Digits        \0\1\2\3\4\5\6\7\8\9
%   Exclamation   \!     Double quote  \"     Hash (number) \#
%   Dollar        \$     Percent       \%     Ampersand     \&
%   Acute accent  \'     Left paren    \(     Right paren   \)
%   Asterisk      \*     Plus          \+     Comma         \,
%   Minus         \-     Point         \.     Solidus       \/
%   Colon         \:     Semicolon     \;     Less than     \<
%   Equals        \=     Greater than  \>     Question mark \?
%   Commercial at \@     Left bracket  \[     Backslash     \\
%   Right bracket \]     Circumflex    \^     Underscore    \_
%   Grave accent  \`     Left brace    \{     Vertical bar  \|
%   Right brace   \}     Tilde         \~}
%
%
%
% \GetFileInfo{caesar.dtx}
% 
%
% \title{The \textsf{caesar} package\thanks{This document
% corresponds to \textsf{caesar}~\fileversion, dated \filedate.}}
% \author{Andy Thomas\\ \texttt{andythomas(at)web.de}}
%
% \maketitle
%
% \changes{v0.10}{2016/05/16}{Initial version, inherited from sidenotes.sty}
% \changes{v0.11}{2016/05/20}{Found some bugs}
% \changes{v0.20}{2020/05/05}{Updated for TeX-Live 2019}
%
% \begin{abstract}
% \noindent This package allows the typesetting of rich content in the margin. 
% It includes text, but also figures and tables,
% which is common in science textbooks such as Feynman's 
% \textit{Lectures on Physics}.
% \end{abstract}
%
% \tableofcontents
%
% \section{Usage}
%
% \DescribeMacro{\sidenote}
% The macro is very similar to the footnote macro and tries to emulate
% its behavior. It just puts the notes in the margin instead of the bottom
% of the page, therefore the name \emph{sidenote}. It has the same parameters 
% as footnote:
% \verb+\sidenote[number]{text}+. All the sidenotes are subsequently 
% numbered and float in the margin to avoid overlap.  
% The optional parameter will manually change the numbering to the given
% value. 
%
% \DescribeMacro{\sidenotemark} 
% Sidenote tries to mimic the footnote behavior and, consequently, provides the 
% same solutions. Sometimes, it is not possible to directly call a sidenote 
% macro, e.g.\ inside of a figure caption. Then, you can use the
% \verb+\sidenotemark[number]+ and \verb+\sidenotetext[number]{text}+ 
% macros. \verb+\sidenotemark+ puts a mark at the current position. Afterwards, 
% outside of the environment that causes the trouble, 
% \DescribeMacro{\sidenotetext}
% it is possible the call \verb+\sidenotetext[number]{text}+ to provide 
% the text and typeset the sidenote. The optional parameter is similar to the 
% sidenote macro and will change the numbering.
%
% \DescribeEnv{marginfigure}
% The marginfigure environment puts a figure and its caption in the margin. 
% Instead of \verb+\begin{figure}[htbp]+ use \verb+\begin{marginfigure}+. 
% The marginfigure has its own caption style named \emph{marginfigure}. 
% Please refer to the \emph{caption} package for styles and their definitions.    
%
% \DescribeEnv{margintable}
% The margintable environment works similar to marginfigure, but with table 
% environments. Use \verb+\begin{margintable}+ instead of 
% \verb+\begin{table}[htbp]+, its caption style is named \emph{margintable}.
%
% \DescribeEnv{figure*}
% The \verb+figure*+ environment is used to position figures across the full 
% page, i.e. the text width plus the margin. The algorithm has to distinguish
% between recto and verso (left and right) pages and might need up to three
% \LaTeX{} runs to provide the desired result. The corresponding caption style
% is called \emph{widefigure}. 
% \DescribeEnv{table*}
% The sister environment for tables is \verb+table*+. Use \emph{widetable} to 
% change its caption style. 
%
% \section{Technical note}
%
% When writing the package, we tried to provide a \emph{minimum} extension to 
% standard \LaTeX{} for typesetting rich content in the margin. This means,
% that there are no sensible default values for most things such as page
% geometry, fonts and font sizes. However, the \emph{caesar\textunderscore 
% book}-class accompanies this package as an example implementation as well 
% as a template we use for our theses. 
%
% Also, this package has less feature than its sister package \emph{sidenotes}.
% It was stripped by the offset functionality and the associated marginnote 
% package completely. The caesar style files are meant to generate different
% output files from the same input file, which does not allow for manual
% intervention. In some sense, caesar tries to completely separate presentation
% and content.
% 
% In addition, we tried to keep compatibility with 
% packages the user might want to use later. However, the following packages 
% are needed by \emph{caesar} and might introduce side effects with other 
% packages.
%
% \section{Required packages}
%
%  \begin{description}
%     \item[caption]
%        allows easier formatting of captions. Please refer to the  
%        \emph{caption} manual for details on styles. 
%     \item[xparse] is used to take advantage of the improved \LaTeX3 syntax.
%        All macros and environments are defined using this package.
%     \item[l3keys2e] provides a key/value mechanism 
%     \item[changepage] is used to correctly shift figure* and table*. It has 
%        to use the option [strict] to work properly. This might lead to an 
%        option clash, if the same package is loaded without this option. 
%  \end{description}%
%
% \section{Implementation}
%
% \iffalse
%<*package>
% \fi
%
% Process the package options: onside 
%
%    \begin{macrocode}
\ExplSyntaxOn 
%
\keys_define:nn { caesar }
  {
    oneside .bool_set:N = \caesar_oneside 
  }
\ProcessKeysOptions { caesar }
%    \end{macrocode}
%
% We need a counter similar to the footnote counter.
%
%    \begin{macrocode}
\newcounter{sidenote} % make a counter 
\setcounter{sidenote}{1} % init the counter 
%    \end{macrocode}


%    \begin{macrocode}
\DeclareExpandableDocumentCommand{\IfNoValueOrEmptyTF}{ m m m }
{
 \IfNoValueTF{#1}
  {#2} 
  {\tl_if_empty:nTF {#1} {#2} {#3}} 
}
%    \end{macrocode}

%
% \noindent Put a marker in the horizontal list to detect subsequent sidenotemarks.
%
%    \begin{macrocode}
\NewDocumentCommand \@caesar@thesidenotemark { m }
{
    \leavevmode 
    \ifhmode 
        \edef \@x@sf {\the \spacefactor }
        \nobreak 
    \fi 
    \hbox {\@textsuperscript {\normalfont #1 }} 
    \ifhmode 
        \spacefactor \@x@sf 
    \fi 
    \relax
}
%
\NewDocumentCommand \@caesar@multisign { } {3sp}
%
\NewDocumentCommand \@caesar@multimarker { } 
{
  \kern-\@caesar@multisign
  \kern\@caesar@multisign\relax 
}
%
\NewDocumentCommand \@caesar@multichecker { } 
{ 
  \dim_compare:nNnTF \lastkern = \@caesar@multisign
  {\@caesar@thesidenotemark{,}}
  {}
}
%
%    \end{macrocode}
%
% \noindent Introduce an internal macro to place the marginal text.
%
%    \begin{macrocode}
\NewDocumentCommand \@caesar@placemarginal { m } 
{ 
    \marginpar{#1}
}
%
%    \end{macrocode}
%
% \begin{macro}{\sidenote}
% 
% Introduce the \verb+\sidenote+ macro with an optional argument.
%
%    \begin{macrocode}
\NewDocumentCommand \sidenote { o +m } 
{ 
  \sidenotemark[#1]
  \sidenotetext[#1]{#2}
  \@caesar@multimarker
}
%    \end{macrocode}
% \end{macro}
%
% \begin{macro}{\sidenotemark}
%
% Sidenotemark is supposed to work similarly to footnotemark.
%
%    \begin{macrocode}
\NewDocumentCommand \sidenotemark { o } 
{ 
  \@caesar@multichecker
  \IfNoValueOrEmptyTF{#1}
    {\@caesar@thesidenotemark{\thesidenote}}
    {\@caesar@thesidenotemark{#1}}
  \@caesar@multimarker
}
%    \end{macrocode}
% \end{macro}
%
% \begin{macro}{\sidenotetext}
%
% Sidenotetext is supposed to work similarly to footnotetext. 
%
%    \begin{macrocode}
\NewDocumentCommand \sidenotetext { o +m } 
{ 
  \IfNoValueOrEmptyTF{#1}
    { 
      \@caesar@placemarginal{\textsuperscript{\thesidenote}{}~#2}
	  \refstepcounter{sidenote}
	}
    {\@caesar@placemarginal{\textsuperscript{#1}~#2}}
}
%    \end{macrocode}
% \end{macro}
%
% \begin{environment}{marginfigure}
%
% The marginfigure environment is similar to the figure environment. But the 
% figure is put in the margin.
%
%    \begin{macrocode}
\newsavebox{\@caesar@marginfigurebox}
\DeclareCaptionStyle{marginfigure}{font=footnotesize}
\NewDocumentEnvironment{marginfigure} { } 
{
  \begin{lrbox}{\@caesar@marginfigurebox}
    \begin{minipage}{\marginparwidth}
      \captionsetup{type=figure,style=marginfigure}
}
{
    \end{minipage}%
  \end{lrbox}%
  \@caesar@placemarginal{\usebox{\@caesar@marginfigurebox}}
}
%    \end{macrocode}
% \end{environment}
%
% \begin{environment}{margintable}
%
% The margintable is similar to the table environment. But the table 
% is put in the margin.
%
%    \begin{macrocode}
\newsavebox{\@caesar@margintablebox}
\DeclareCaptionStyle{margintable}{font=footnotesize}
\NewDocumentEnvironment{margintable} { }
{
  \begin{lrbox}{\@caesar@margintablebox}
    \begin{minipage}{\marginparwidth}
      \captionsetup{type=table,style=margintable}
}
{
    \end{minipage}
  \end{lrbox}
  \@caesar@placemarginal{\usebox{\@caesar@margintablebox}}
}
%    \end{macrocode}
% \end{environment}
%
% \begin{environment}{figure*}
%
% The figure* environment provides a figure environment for figures that 
% span across the full page (text plus margin width).
%
%    \begin{macrocode}  
\AtBeginDocument{%
\newlength{\@caesar@extrawidth}
\setlength{\@caesar@extrawidth}{\marginparwidth}
\addtolength{\@caesar@extrawidth}{\marginparsep}}
%
\NewDocumentEnvironment{autoadjustwidth}{ m m }%
{
    \bool_if:NTF \caesar_oneside
    {
        \begin{adjustwidth}{#1}{#2}
    }
    {
        \begin{adjustwidth*}{#1}{#2}
    }
}
{
    \bool_if:NTF \caesar_oneside
    {
        \end{adjustwidth}
    }
    {
        \end{adjustwidth*}
    }
}
%
\DeclareCaptionStyle{widefigure}{font=footnotesize}
\RenewDocumentEnvironment{figure*}{ O{htbp} }
{
    \begin{figure}[#1]
        \begin{autoadjustwidth}{}{-\@caesar@extrawidth}
        \captionsetup{style=widefigure}
}
{
        \end{autoadjustwidth}
    \end{figure}
}
%    \end{macrocode}
% \end{environment}
%
% \begin{environment}{table*}
%
% The table* environment provides a table environment for figures across 
% text and margin width.
%
%    \begin{macrocode}  
\DeclareCaptionStyle{widetable}{font=footnotesize}
\RenewDocumentEnvironment{table*}{ O{htbp} }
{
    \begin{table}[#1]
        \begin{autoadjustwidth}{}{-\@caesar@extrawidth}
        \captionsetup{style=widetable}
}
{
        \end{autoadjustwidth}
    \end{table}
}
\ExplSyntaxOff
%    \end{macrocode}
% \end{environment}
% \Finale
\endinput
