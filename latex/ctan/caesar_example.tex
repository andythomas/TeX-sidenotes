\documentclass{caesar_book}
%
% !BIB TS-program = biber
%
% -- use dummy graphics
\usepackage{mwe}

%auto generate the bib file
\usepackage{filecontents}
%
\begin{filecontents}{\jobname.bib}
@book{Tufte2006,
 author = {Edward R. Tufte},
 title = {Beautiful Evidence},
 year = {2006},
 publisher = {Graphics Press, {LLC}},
 isbn = {0-9613921-7-7}
}

@book{Tufte1990,
 author = {Edward R. Tufte},
 title = {Envisioning Information},
 publisher = {Graphics Press},
 year = {1990},
 isbn = {0-9613921-1-8}
}
\end{filecontents}

% -- language: English --
%
\usepackage{csquotes}
\usepackage[english]{babel}
% -- biblatex --
\usepackage[backend=biber,style=philosophy-classic]{biblatex} % xxx
% the .bib file with the references
\addbibresource{\jobname.bib} 

% Information about the book
% to be put on the title page
\title{Caesar Quick start}
\author{Andy Thomas}
\publisher{Bielefeld University}

\begin{document}
% no page numbering in front matter
\frontmatter
% generate the title page
\maketitlepage
% show the table of contents
\tableofcontents
% start to number the pages
\mainmatter
% The first chapter with annotation and citations
\chapter{Quick start}
%
We compiled a minimal example file to show the basic use of the caesar class, which allows the typesetting of (science) textbooks or theses. The class itself is a reference implementation of the \emph{sidenotes} package. The package provides the additional functionality\sidenote{namely the use of marginal material such as this note or even figures or tables.} and the class gives sensible default values for page margins, chapter formatting and such. The caesar class is derived from the standard \LaTeX-book class and a little bit of experience with the standard class might be very helpful. In this example, biblatex is used for the references.

The first pages of the book (the frontmatter) are not numbered, the numbering starts after the \texttt{mainmatter} macro, which is called after the generation of the title page. The layout has ample margins to allow annotations. A main feature and the package is the sidenote, which is a footnote in the margin and can be placed with the \texttt{sidenote} macro.\sidenote{All information is on the same page, no turning of pages is necessary.} It is very similar to \texttt{footnote} and tries to emulate its behavior. The sidenote moves up or down (floats) to not overlap with other floats in the margin and all the sidenotes are subsequently numbered. 

References can be put in the margin as well.\sidecite[For the ideas behind all this, please see:][ and more work by Tufte.]{Tufte1990,Tufte2006} The macro was named \texttt{sidecite} and is defined with two optional parameters (prefix and postfix) similar to \texttt{cite} taken from the biblatex package. The next two sections describe the different options for the use of figures and tables in a document. We start with a couple of figures.
% A section with a couple of figures
\section{Figures}
%
\begin{marginfigure}%
    \includegraphics[width=\marginparwidth]{example-image-a}%
    \caption{A small rectangle put in the margin.\label{rectangle}}%
\end{marginfigure}%
%
There are three basic options to include figures in a document. The first option is a small figure and its caption in the margin. Figure \ref{rectangle} shows that with a gray rectangle framing the letter \emph{A}. We simply use the \texttt{marginfigure} environment instead of the \texttt{figure} one. 

The next alternative is a figure in the text frame. The figure is placed using the regular \LaTeX-figure environment and its  caption, which is displayed in figure~\ref{rectangle2}. 
%
\begin{figure}[htbp]%
	\includegraphics[height=180pt,width=\textwidth]{example-image-b}%
	\caption{A larger rectangle in the main area of the text, i.e.\ it does not span into the margin.}%
	\label{rectangle2}%
\end{figure}%
%

In case that a wider figure is needed, the third option spans over the text as well as the margin area. Here, the common \texttt{figure*} environment can be used. The figure options make it easy to choose the appropriate size for a given input file. 
%
\begin{figure*}[htbp]
    \includegraphics[height=180pt,width=400pt]{example-image-c}%
    \caption{An even larger rectangle. This is the widest figure option. Both, the text as well as the margin width are used for the diagram.}
    \label{rectangle3}
\end{figure*}
%

% Next section with a variety of tables
\section{Tables}
The same set of options (small, normal and wide) are also available for tables. The first option is a small table in the margin, this \texttt{margintable} is shown in table \ref{table1}.
%
\begin{margintable}%
	\begin{tabular}{lll}%
     A&B&C\\%
     0.50&0.47&0.48\\%
    \end{tabular}%
	\vspace{2pt}
	\caption{A couple of numbers in a table in the margin.\label{table1}}%
\end{margintable}%

Table \ref{table2} displays a larger table with more numbers. This is done using regular \LaTeX-macros for placing the table along with its caption. 
%
\begin{table}[htbp]%
	 \begin{tabular}{lllllllll}%
     A&B&C&D&E&F&G&H&I\\%
    0.50&0.47&0.48&0.50&0.47&0.48&0.60&0.39&1.00\\%
    \end{tabular}%
	\vspace{2pt}%
	\captionsetup{width=\textwidth, justification=justified}%
	\caption{A couple of numbers in a larger table. This table spans the usual text width.\label{table2}}%
\end{table}%

The \texttt{table*} environment is also defined in analogy to \texttt{figure*} and is demonstrated in table \ref{table3}.
%
\begin{table*}[h!]
 \begin{tabular}{lllllllllllll}%
     A&B&C&D&E&F&G&H&I&J&K&L&M\\%
    0.50&0.47&0.48&0.50&0.47&0.48&0.60&0.39&1.00&0.50&0.47&0.48&0.60\\%
  \end{tabular}%
  \vspace{2pt}
  \caption{Even more numbers in a big table are shown here. This table spans across the full page, text width plus margin.\label{table3}}%
\end{table*}

%
\section{More information}
\marginpar{It is also possible to put a comment in the margin without a corresponding mark in the text with \texttt{marginpar}.} This is a short example file to show the features of the caesar class together with the sidenotes package. Sometimes it is necessary to compile the document up to 3 times in order to get the alignment of all objects correctly.
%
\printbibliography[heading=bibintoc]
 %
\end{document}
